\chapter{Технологическая часть}

\section{Средства реализации}

Для реализации ПО был выбран язык C++ \cite{C++}.
В данном языке есть все требующиеся инструменты для данной лабораторной работы.
В качестве среды разработки была выбрана среда VS code \cite{vscode}.

\section{Реализация алгоритма}

Реализация кодирования LZW.

\begin{lstlisting}
void compressInternal(const std::vector<uint8_t>& inputFile)
{
    initializeDictionary();

    std::vector<uint8_t> currentSubsequence;
    int nextIndex = 0;
    uint8_t nextByte;
    TrieNode* currentNode = rootNode;

    int code = 0xFF + 1;

    while (nextIndex < inputFile.size())
    {
        nextByte = inputFile[nextIndex];
        if (currentNode->children.contains(nextByte))
        {
            currentNode = &currentNode->children[nextByte];
            nextIndex++;
        }
        else
        {
            tempOut->emplace_back(currentNode->code, getBitsToRepresentInteger(code));
            currentNode->children[nextByte].code = code;
            code++;
            currentNode = rootNode;
        }
    }

    if (currentNode != rootNode)
    {
        tempOut->emplace_back(currentNode->code, getBitsToRepresentInteger(code));
    }

    std::cout << "dict size: " << getDictSize() << std::endl;
}
\end{lstlisting}

\section{Тестирование ПО}

В таблице \ref{tbl:functional_test} приведены тесты для алгоритма шифрования LZW. 
Применена методология черного ящика. Тесты пройдены \textit{успешно}.

\begin{table}[ht!]
	\begin{center}
		\captionsetup{justification=raggedright,singlelinecheck=off}
		\caption{\label{tbl:functional_test} Функциональные тесты}
		\begin{tabular}{|c|c|c|}
			\hline
			Входная строка & Размер входной (Байт) & Размер выходной (Байт) \\ 
			\hline
			$aba$          & 3  & 4 \\
			$abaaba$       & 6  & 6 \\
			<<>>           & 0  & 0 \\
            $abaabaabaaba$ & 12 & 8 \\
			\hline
		\end{tabular}
	\end{center}
\end{table}
