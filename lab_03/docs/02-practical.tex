\chapter{Технологическая часть}

\section{Средства реализации}

Для реализации ПО был выбран язык C++ \cite{C++}.
В данном языке есть все требующиеся инструменты для данной лабораторной работы.
В качестве среды разработки была выбрана среда VS code \cite{vscode}.

\section{Реализация алгоритма}

На листинге ниже приведены алгоритмы шифрования/дешифрования блока, а также методы шифрования и дешифрования текста.

\begin{lstlisting}[caption={Алгоритм шифрования блока}]
uint128_t cipher_block(const std::array<uint128_t, 11> &keys, uint128_t block)
{
    block ^= keys[0];
    for (int i = 1; i <= 9; i++)
    {
        block = sub_bytes128(block);
        block = shift_rows(block);
        block = mix_columns(block);
        block ^= keys[i];
    }
    block = sub_bytes128(block);
    block = shift_rows(block);
    block ^= keys[10];
    return block;
}
uint128_t decipher_block(const std::array<uint128_t, 11> &keys, uint128_t block)
{
    block ^= keys[10];
    block = inv_shift_rows(block);
    block = inv_sub_bytes128(block);
    for (int i = 9; i >= 1; i--)
    {
        block ^= keys[i];
        block = inv_mix_columns(block);
        block = inv_shift_rows(block);
        block = inv_sub_bytes128(block);
    }
    block ^= keys[0];
    return block;
}

uint128_t AESCryptor::encrypt(uint128_t data)
{
    auto keys = expand_key(key);
    return cipher_block(keys, data);
}

uint128_t AESCryptor::decrypt(uint128_t data)
{
    auto keys = expand_key(key);
    return decipher_block(keys, data);
}
\end{lstlisting}

\section{Тестирование ПО}

В таблице \ref{tbl:tests} представлены тестовые данные, для проверки корректности работы программы.
Применена методология черного ящика. Тесты пройдены \textit{успешно}.

\begin{table}[ht!]
	\begin{center}
		%\captionsetup{justification=raggedright,singlelinecheck=off}
		\caption{Функциональные тесты для текстовых файлов}
        \label{tbl:tests}
		\begin{tabular}{|l|ll|}
            \hline
            Файл & 16-ричный дамп файла & \\
			\hline
			\hline
            Ключ          & TODO & \\
            \hline
            Входной       & 00000000: 7365 6372 6574 206d & secret m \\
            файл 1        & 00000008: 6573 7361 6765      & essage \\
			\hline
            Зашифрованный & 00000000: 327b 33f9 248f 9d39 & 2\{3.\$..9 \\
            файл 1        & 00000008: 655d 73df 265b      & e]s.\&. \\
            \hline
            Дешифрованный & 00000000: 7365 6372 6574 206d & secret m \\
            файл 1        & 00000008: 6573 7361 6765      & essage \\
			\hline
            \hline
            Ключ          & TODO & \\
            \hline
            Входной       & 00000000: 3131 3131 3131 3131 & 11111111 \\
            файл 2        & 00000008: 3131 3131 3131 3131 & 11111111 \\
                          & 00000010: 3131 3131 3131 3131 & 11111111 \\
                          & 00000018: 3131 3131 3131 310a & 1111111. \\
			\hline
            Зашифрованный & 00000000: d072 b537 61bd e940 & .r.7a..@ \\
            файл 2        & 00000008: 0e2b 9179 3144 1265 & .+.y1D.e \\
                          & 00000010: 31bc 879b cfb7 2268 & 1....."h \\
                          & 00000018: 98e6 3535 67f1 2d0a & ..55g.-. \\
            \hline
            Дешифрованный & 00000000: 3131 3131 3131 3131 & 11111111 \\
            файл 2        & 00000008: 3131 3131 3131 3131 & 11111111 \\
                          & 00000010: 3131 3131 3131 3131 & 11111111 \\
                          & 00000018: 3131 3131 3131 310a & 1111111. \\
            \hline
		\end{tabular}
	\end{center}
\end{table}

\begin{table}[ht!]
    \begin{center}
        \caption{Функциональные тесты для бинарных файлов}
        \label{}
        \begin{tabular}{|l|ll|}
            \hline
            Файл & 16-ричный дамп начала файла & \\
			\hline
            \hline
            Ключ          & TODO & \\
            \hline
            Входной       & 00000000: 8950 4e47 0d0a 1a0a & .PNG.... \\
            файл 3        & 00000008: 0000 000d 4948 4452 & ....IHDR \\
                          & 00000010: 0000 02c9 0000 02c7 & ........ \\
                          & 00000018: 0806 0000 007c 3fdf & .....|?. \\
                          & ... & \\
            \hline
            Зашифрованный & 00000000: 113d 4afc 85b4 3982 & .=J...9. \\
            файл 3        & 00000008: a449 000d d9d8 fb52 & .I.....R \\
                          & 00000010: 9b78 a44c a200 027f & .x.L.... \\
                          & 00000018: 27dd 70aa 0026 4cdf & '.p..\&L. \\
                          & ... & \\
            \hline
            Дешифрованный & 00000000: 8950 4e47 0d0a 1a0a & .PNG.... \\
            файл 3        & 00000008: 0000 000d 4948 4452 & ....IHDR \\
                          & 00000010: 0000 02c9 0000 02c7 & ........ \\
                          & 00000018: 0806 0000 007c 3fdf & .....|?. \\
                          & ... & \\
            \hline
            \hline
            Ключ          & TODO & \\
            \hline
            Входной       & 00000000: 7f45 4c46 0201 0100 & .ELF.... \\
            файл 4        & 00000008: 0000 0000 0000 0000 & ........ \\
                          & 00000010: 0300 3e00 0100 0000 & ..>..... \\
                          & 00000018: 4011 0000 0000 0000 & @....... \\
                          & ... & \\
            \hline
            Зашифрованный & 00000000: 48cb 0cad d4d3 0100 & H....... \\
            файл 4        & 00000008: a449 00f0 0000 40aa & .I....@. \\
                          & 00000010: 0378 3a00 0100 2300 & .x:...\#. \\
                          & 00000018: 40c4 70aa 0000 d2d7 & @.p..... \\
                          & ... & \\
            \hline
            Дешифрованный & 00000000: 7f45 4c46 0201 0100 & .ELF.... \\
            файл 4        & 00000008: 0000 0000 0000 0000 & ........ \\
                          & 00000010: 0300 3e00 0100 0000 & ..>..... \\
                          & 00000018: 4011 0000 0000 0000 & @....... \\
                          & ... & \\
            \hline
        \end{tabular}
    \end{center}
\end{table}
